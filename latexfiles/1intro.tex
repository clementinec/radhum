%!TEX root = _all.tex
% Energy savings of alternative energy systems/delivery systems.
Buildings are prominent energy end-users, and are often considered to be the easier target when attempting to achieve energy savings or reducing carbon emissions. The energy usage of the heating and cooling systems is also identified as one of the most accessible areas to achieve significant energy savings\cite{balaras_european_2007}. As there are many different types of heating and cooling systems that are widely available, we want to investigate particularly the implications of radiant systems in contrast to all-air systems since the contrast would be larger\cite{stetiu_energy_1999}. This can be explained by how the radiant systems do not need to explicitly condition the air temperature, which will result in smaller heating and cooling load since water and other working fluids are better heat carriers\cite{moe_thermally_2010}. In the meantime, radiant systems appear to have comparable \cite{karmann_thermal_2017} if not better thermal comfort implications according to existing studies \cite{fabrizio_numerical_2012}, hence forming good comparison to the prevailing air-based heating/cooling systems in the US.

%The relationship to relative humidity.
Unlike air temperature, or the targeted fluid/surface temperature that are often used in the control of air or radiant systems, relative humidity (RH) is a much less commonly observed control variable\cite{rentel-gomez_decoupled_2001}. Relative-humidity-based control is currently used in more strictly-controlled environments such as piano rooms, archival storage rooms and specialized medical facilities\cite{staniforth_appropriate_1994,han_experimental_2011}. RH-based control is currently very expensive for residential or large commercial projects to consider\cite{henderson_energy_2014}. This is not only because the demand of RH-specific environments is less common, but also because the value of RH is codependent on the condition of the humid air condition, and cannot be easily controlled through a feedback control loop\cite{alahmer_effect_2011}. 

%Air as we know it
The moist air in a room is a mixture of both dry air and water vapor. Dry air has properties that are very close to ideal gas while the water vapor is ideal gas, we model their mixture using the ideal gas law and can estimate this mixture’s remaining properties (pressure, humidity ratio and enthalpy) when a specific set of property (temperature and RH, in this example) is given\cite{ansi/ashrae_standard_2017}. Graphically, this is also often explained by the psychrometric chart published by ASHRAE \cite{american_society_of_heating_refrigerating_and_air-conditioning_engineers_ashrae_1972}. However, the actual psychrometric processes that air conditioners undergoes is a myth to most users, since the feedback variable that most of their systems would allow them to control is only the air temperature\cite{pasgianos_nonlinear_2003}. Although ASHRAE does recommend RH to be maintained between 30\% to 60\%, adding humidity to the air is still a secondary concern when people are purchasing and installing home air conditioning systems\cite{american_society_of_heating_2007_2007}. 

%RH – dry and wet
According to some latest research, the health implications of RH can be significant, although often in a delayed and indirect manner. Reports on RH below 20\% could cause eye irritation and dryness/stuffiness complaints, and ultra-low RHs below that may even cause eye and air pathway and skin symptoms. RH below 10\% could even desiccate the mucous membrane over time and lead to further agitation with both the eye and nasal cavity\cite{sunwoo_physiological_2006-1}. Conversely, higher RH levels also have the potential to deteriorate building materials and grow mold that may lead to respiratory symptoms among building occupants. However, the RH that would allow rapid growth of bacteria and viruses are often found to be at least 75\%, which is much higher than the mandate from ASHRAE or ISO 7730 (30\% to 70\%). There is, ultimately, very little regulation or guidelines regarding what the RH inside a room should be kept at, and anything above 30\% and below 60\% is acceptable. This is potentially also the reason why RH is much less controlled in conventional systems.

%Relationship to Energy
Existing literature have already established the link between energy savings and expanded set points of air\cite{hoyt_extending_2015}. The verified energy potentials were demonstrated through air-based systems in simulated building cases through different climates. This included both the sensible and latent heat following the underlying hypothesis of the building case used and the occupancy density that was selected. Using radiant systems may prevent the excessive energy usage needed to remove the latent heat in the air in the summer. The relationship between the sensible and latent cooling load is less often studied but has seen some recent advances: new concepts such as the enthalpy degree days \cite{krese_incorporation_2011} and latent enthalpy days were proposed in contrast to the cooling degree day method used in estimating annual cooling loads\cite{huang_climatic_1986}.

%What we want to do with this study  
In this study, we hope to take advantage of the existing literature to understand the benefit of further expanding the set point temperatures through the usage of radiant systems. This would include discussions on the energy savings from preventing the latent loads that would otherwise have occurred when using all-air systems, as well as the improvement of RH without additional humidifiers for the heating season.  To better illustrate the energy saving potentials and the corresponding improvement in RH, we will run the analysis through hypothetical household located across the 48 continental states in the US to illustrate the energy and humidity potentials that can be found across different areas. 
