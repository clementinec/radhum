The relative humidity improvements that we identified carries some very interesting weight in furthering our evaluation of different system types. Traditionally, radiant cooling systems are only used when the relative humidity and dew point temperatures of an environment does not pose a threat to condense, otherwise the system will need to be carefully engineered to avoid condensation problems. Radiant heating systems (primarily radiators and radiant floor heating) primarily occurs in larger multi-family or high-rise residential buildings when district-level heating is possible. Although radiant heating systems were repeatedly suggested by researchers to provide better thermal comfort in comparison to air-based heating systems, some recent studies have argued against it. This improvement in relative humidity due to reduced set point temperature in radiant systems has, insofar, yet to be a major topic in the field of building engineering.

There are a number of possible answers to its absence. The first being many of the existing radiant heating systems uses 75 $\degree F$ (23.89 $\degree C$) air temperature as the feedback control variable of radiant systems. This dry-bulb air temperature is meant to ensure the occupant comfort situation inside the conditioned space. However, radiant systems are fundamentally responsible for providing a satisfying radiant environment that ensures satisfiable heat dissipation of the occupants despite the heating loads from the envelope and other miscelleneous equipments. Radiant systems are not meant to serve as heat exchangers that heats up the indoor air and maintain it at 75 $\degree F$. The authors have recently done some research with respect to the possibility of separating the radiant sensing from the air temperature sensing such that the two modes of heat exchange can be viewed separately, but there has yet to be adequate research or technology in the market that allows such separation. Up to now, it is still very common to equate thermal comfort as operative temperature as measurement of globe temperature as air temperature. We believe it is important to emphasize their differences and argue for further improvements in the state-of-the-art technology in the field.

