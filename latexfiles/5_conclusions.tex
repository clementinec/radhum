We examined the heating and cooling of the humid air across continental US and analyzed the energy savings potential that can be expected when alternative set points are used. A highlight of this study is how the heating potential is analyzed and the energy savings analyzed for better understanding of the potential energy savings of radiant systems. Beginning with pure psychrometric analysis, we use the humid air properties of the air to determine the amount of heating and cooling that is necessary to achieve a desirable state for the humid air. This analysis is independent of a specific type of air handling and a predetermined process of a given system, and is generically applicable for large-scale analysis. To do so, we used the weather data files made available by the NOAA ISD-lite project, and estimated the overall heating degree days and cooling degree days against its enthalpy degree days to further characterize the latent loads that occur for different cities in different climate zones. We obtained results that shows the energy savings potential for both the heating and the cooling scenarios across the country.

We hope the results from this study can help further investigations being directed at the latent loads in buildings that are caused by using all-air systems for both the heating and the cooling scenarios. Although we have demonstrated that there are energy savings potentials through expanding the set point temperature for the indoor environment, the scope of investigation of this paper does not extend to specific systems or components, and should therefore should not be interpreted as explicit energy savings for building systems. We hope to inspire future investigations on the cooling and heating capabilities of such systems, and observe improved system designs as a result of this study.