%!TEX root = _all.tex

Alternatively, the energy savings that we have been able to identify for both the heating and the cooling are also of different orders of magnitude. Comparing the order of magnitude of the energy that needs to be added/removed from the air shown in Figure~\ref{fg:heatall} and ~\ref{fg:coolall}, the latent heat that needs to be removed could be up to 10 times larger than that of the humidity requires adding, which speaks to the order of magnitude of the energy demand from de-humidification processes. However, it is equally important to stress that our discussion within the scope of this paper is purely driven by the psychrometric process.

More explicitly speaking, the latent component of the heating and the cooling scenarios are fundamentally different. For the heating scenario, the latent load is the humidity that needs to be added by humidifiers to maintain satisfying level of humidity (50\% in this study). For the cooling scenario, the latent load we were comparing was obtained by separating the sensible from the total cooling demand resulting from outdoor air. Therefore, for the heating scenario, the energy that needs to be added to the air through humidifiers cannot be directly translated into energy demand. For the cooling scenario, the the differences between the latent energy are in fact the energy demands that can be reduced through selection of different set points.

However, regarding the humidity removal process,it's also important to point out that neither of the scenarios can be directly translated into energy savings as there may potentially be different systems invovled in achieving the desirable state of humid air. This allowed us to avoid focusing on a specific type of building system and its corresponding component effiiencies within the scope of this analysis. Further discussions regarding the benefits of selecting certain humidity removal technologies, such as the chilled water cooling/heating system, dual (dessicant and enthalpy) wheel or direct exchange cooling coils. 
