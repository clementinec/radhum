Hoping to create a generic building energy footprint for comparison across the United States, we are assuming an average household with average square footage, household member counts, and energy consumption profiles that are made publicly available by the Department of Energy and US Census. For the purpose of this study, we consider only the residential conditioning systems. For this purpose, we assume the size of the household to be 2.52 occupants (cite US. Census.), which requires no more than 37.8 CFM of fresh air according to ASHRAE. In order to estimate the required fresh air rate, we assume the size of the house through the average size of buildings built (cite DOE?) as 2392 squared feet. As according to ASHRAE Standard 62, the newly-built buildings are instructed to achieve an infiltration rate of 2 CFM per every 100 squared feet, this allows us to estimate the ventilation rate of the hypothetical average house to have 47.84 CFM through infiltration. This is above the required fresh air amount for 2.52 hypothetical occupants in the average household in the United States. Assuming the height of the houses to be 9 feet, we obtained a fresh air percentage of 18 per cent, which allowed us to estimate the condition of the mixed air with a targeted indoor air condition and outdoor air condition.  The set points and the psychrometric processes we are assuming are introduced separated in the subsequent sections.

